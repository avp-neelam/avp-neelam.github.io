\documentclass{article}
\usepackage[utf8]{inputenc}
\usepackage{mathtools}
\usepackage{stmaryrd}
\usepackage[alphabetic]{amsrefs}
\usepackage{amsfonts, amsmath, amsthm, amssymb}
\usepackage[margin=1.5in]{geometry}
\usepackage[colorlinks]{hyperref}
\usepackage[table]{xcolor}
\usepackage{tikz}\usetikzlibrary{cd}
\usepackage{enumerate}

% env style
\theoremstyle{plain}
\newtheorem{thm}{Theorem}[section]
\newtheorem{lem}[thm]{Lemma}
\newtheorem{cor}[thm]{Corollary}
\newtheorem{prop}[thm]{Proposition}
\theoremstyle{definition}
\newtheorem{definition}{Definition}[section]
\newtheorem{conjecture}[thm]{Conjecture}
\newtheorem{prob}{Problem}
\newtheorem{example}[definition]{Example}
\newtheorem{xca}[definition]{Exercise}
\theoremstyle{remark}
\newtheorem{remark}[thm]{\bf Remark}
\newtheorem{tab}[thm]{\bf Table}
\newtheorem*{note}{Note}

\numberwithin{equation}{section}

% custom macros
\def\R{{\mathbb R}}
\def\N{{\mathbb N}}
\def\C{{\mathbb C}}
\def\Q{{\mathbb Q}}
\def\k{k}
\def\O{\mathcal{O}}
\def\sheaf#1{\mathcal{#1}}

\def\GAP{\textsc{Gap}}
\def\gap{\color{blue}gap> }
\def\QPA{\textsc{Qpa}}
\def\Magma{\textsc{Magma}}

\newcommand{\cat}{{\mathcal C}}
\newcommand{\Ob}{\text{Ob}}
\DeclareMathOperator{\id}{id}
\DeclareMathOperator{\coker}{coker}
\renewcommand{\hom}{\textup{Hom}}
\DeclareMathOperator{\ext}{Ext}
\DeclareMathOperator{\im}{im}
\DeclareMathOperator{\rep}{rep}
\DeclareMathOperator{\SET}{\mathbf{Set}}
\DeclareMathOperator{\CAT}{\mathbf{Cat}}
\DeclareMathOperator{\QUIV}{\mathbf{Quiv}}
\newcommand{\dimvec}[3]{\begin{smallmatrix}#1\\#2\\#3\end{smallmatrix}}

\title{MATH 299Q: Homework 1\\{\Large Quiver Representations}}
\author{}
\date{}

\begin{document}

\maketitle

Let $\k$ denote an algebraically-closed field.

\begin{enumerate}[1.]
    \item Suppose $V$ and $W$ are two finite-dimensional $\k$-vector spaces, with $\dim V=n$ and $\dim W=m$. Let $T\in\hom(V,W)$, that is $T:V\to W$.
    \begin{enumerate}[(a)]
        \item Define $\im T$ and $\ker T$.
        \item What does it mean for $T$ to be \emph{injective}? What about \emph{surjective}? What conditions on $n$ and $m$ must we have for $T$ to be an \emph{isomorphism}?
        \item Show that $\hom(V, W)$ has a natural $\k$-vector space structure.\newline [\emph{Hint: For $f,g\in\hom(V,W)$ show that $f+\lambda g\in\hom(V,W)$ for $\lambda\in\k$.}]
    \end{enumerate}
    \vfill
    \item 
    \begin{enumerate}
        \item Draw your favorite (finite) quiver!
        \item Give a non-trivial representation of the quiver you drew in (a).
    \end{enumerate}
    \vfill
    \item Let $Q$ be the quiver
    \[\begin{tikzcd}
        1\arrow[rrd, "\alpha"] && && \\
        && 3 && 4\arrow[ll, "\gamma"'] \\
        2\arrow[rru, "\beta"'] && &&
    \end{tikzcd}\]
    and consider the representations $M$ and $N$:
    \[M:\begin{tikzcd} \k\arrow[rrd, "\begin{bmatrix}1\\0\end{bmatrix}"] && && \\ && \k^2 && \k\arrow[ll, "\begin{bmatrix}1\\1\end{bmatrix}"'] \\ \k\arrow[rru, "\begin{bmatrix}0\\1\end{bmatrix}"'] && &&\end{tikzcd}
    \qquad\qquad
    N:\begin{tikzcd}\k\arrow[rrd, "1"] && && \\ && \k && \k\arrow[ll, "1"'] \\ \k\arrow[rru, "1"'] && &&\end{tikzcd}\]
    Prove that $\hom(M, N)\cong\k^2$. [\emph{Hint: Draw a morphism between $M$ and $N$.}]
    \vfill
\end{enumerate}

\end{document}
