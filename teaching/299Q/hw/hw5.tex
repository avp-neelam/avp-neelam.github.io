\documentclass{article}
\usepackage[utf8]{inputenc}
\usepackage{mathtools}
\usepackage{stmaryrd}
\usepackage[alphabetic]{amsrefs}
\usepackage{amsfonts, amsmath, amsthm, amssymb}
\usepackage[margin=1.5in]{geometry}
\usepackage[colorlinks]{hyperref}
\usepackage[table]{xcolor}
\usepackage{tikz}\usetikzlibrary{cd}
\usepackage{enumerate}

% env style
\theoremstyle{plain}
\newtheorem{thm}{Theorem}[section]
\newtheorem*{thm*}{Theorem}
\newtheorem{lem}[thm]{Lemma}
\newtheorem{cor}[thm]{Corollary}
\newtheorem{prop}[thm]{Proposition}
\theoremstyle{definition}
\newtheorem{definition}{Definition}[section]
\newtheorem{conjecture}[thm]{Conjecture}
\newtheorem{prob}{Problem}
\newtheorem{example}[definition]{Example}
\newtheorem{xca}[definition]{Exercise}
\theoremstyle{remark}
\newtheorem{remark}[thm]{\bf Remark}
\newtheorem{tab}[thm]{\bf Table}
\newtheorem*{note}{Note}

\numberwithin{equation}{section}

% custom macros
\def\Z{{\mathbb Z}}
\def\R{{\mathbb R}}
\def\N{{\mathbb N}}
\def\C{{\mathbb C}}
\def\Q{{\mathbb Q}}
\def\k{k}
\def\O{\mathcal{O}}
\def\sheaf#1{\mathcal{#1}}
\def\S{\mathcal S}
\def\P{\mathcal P}
\def\I{\mathcal I}

\def\GAP{\textsc{Gap}}
\def\gap{\color{blue}gap> }
\def\QPA{\textsc{Qpa}}
\def\Magma{\textsc{Magma}}

\newcommand{\cat}{{\mathcal C}}
\newcommand{\Ob}{\text{Ob}}
\DeclareMathOperator{\id}{id}
\DeclareMathOperator{\coker}{coker}
\renewcommand{\hom}{\textup{Hom}}
\DeclareMathOperator{\End}{End}
\DeclareMathOperator{\ext}{Ext}
\DeclareMathOperator{\im}{im}
\DeclareMathOperator{\rep}{rep}
\DeclareMathOperator{\SET}{\mathbf{Set}}
\DeclareMathOperator{\CAT}{\mathbf{Cat}}
\DeclareMathOperator{\QUIV}{\mathbf{Quiv}}
\newcommand{\dimvec}[3]{\begin{smallmatrix}#1\\#2\\#3\end{smallmatrix}}


\title{MATH 299Q: Homework 5\\{\Large Quiver Representations}}
\author{}
\date{}

\begin{document}

\maketitle

Let $\k$ be an algebraically closed field.

\begin{enumerate}[1.]
    \item Recall Homework 3, which asked you to compute the AR-quiver for a quiver $Q$ with $\Delta_Q=\overrightarrow{\mathbb{A}_3}$ and generalize this pattern to $\Delta_Q=\overrightarrow{\mathbb{A}_n}$.
    
    \qquad Using the solutions to that problem, derive a formula for the number of indecomposable representations in $\rep Q$ for $\Delta_Q=\mathbb{A}_n$ for any arrow orientation. i.e., find the number of vertices in the AR-quiver of $Q$ in terms of $n=|Q_0|$.

    \emph{[Hint: Consider the construction of AR-quiver that utilized regular $n+3$-gons.]}
    
    \vfill
    \item Let $Q$ be a connected quiver without orientated cycles with $n$ vertices, fix some $\mathbf{d}=(d_i)\in\Z_{\geq 0}^n$ and define the space of all representations $M\in\rep Q$ with dimension vector $E_{\mathbf{d}} := \{M\in\rep Q\mid\underline{\dim}M=\mathbf{d}\}$. It should be clear that $E_{\mathbf{d}}$ is a $\k$-vector space.
    
    \qquad Define the group
    \[G_{\mathbf{d}} := \prod_{i\in Q_0}\textup{GL}_{d_i}(\k).\]
    This group acts naturally on $E_{\mathbf{d}}$ via conjugation; if $g=(g_i)\in G_{\mathbf{d}}$, $M=(M_i,\varphi_\alpha)\in E_{\mathbf{d}}$, and $i\xrightarrow{\alpha}j\in Q_1$, then $(g\cdot \varphi)_\alpha = g_j\varphi_\alpha g_i^{-1}$. We denote the orbit of $M\in E_{\mathbf{d}}$ under $G_{\mathbf{d}}$ by $\O_M := \{g\cdot M\mid g\in G_{\mathbf{d}}$. Show that the orbit $\O_M$ is the isoclass of the representation $M$, i.e., $\O_M = \{M'\in\rep Q\mid M\cong M'\}$.
    \vfill
\end{enumerate}

\end{document}
