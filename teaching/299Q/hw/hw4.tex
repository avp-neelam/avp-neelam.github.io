\documentclass{article}
\usepackage[utf8]{inputenc}
\usepackage{mathtools}
\usepackage{stmaryrd}
\usepackage[alphabetic]{amsrefs}
\usepackage{amsfonts, amsmath, amsthm, amssymb}
\usepackage[margin=1.5in]{geometry}
\usepackage[colorlinks]{hyperref}
\usepackage[table]{xcolor}
\usepackage{tikz}\usetikzlibrary{cd}
\usepackage{enumerate}

% env style
\theoremstyle{plain}
\newtheorem{thm}{Theorem}[section]
\newtheorem*{thm*}{Theorem}
\newtheorem{lem}[thm]{Lemma}
\newtheorem{cor}[thm]{Corollary}
\newtheorem{prop}[thm]{Proposition}
\theoremstyle{definition}
\newtheorem{definition}{Definition}[section]
\newtheorem{conjecture}[thm]{Conjecture}
\newtheorem{prob}{Problem}
\newtheorem{example}[definition]{Example}
\newtheorem{xca}[definition]{Exercise}
\theoremstyle{remark}
\newtheorem{remark}[thm]{\bf Remark}
\newtheorem{tab}[thm]{\bf Table}
\newtheorem*{note}{Note}

\numberwithin{equation}{section}

% custom macros
\def\R{{\mathbb R}}
\def\N{{\mathbb N}}
\def\C{{\mathbb C}}
\def\Q{{\mathbb Q}}
\def\k{k}
\def\O{\mathcal{O}}
\def\sheaf#1{\mathcal{#1}}
\def\S{\mathcal S}
\def\P{\mathcal P}
\def\I{\mathcal I}

\def\GAP{\textsc{Gap}}
\def\gap{\color{blue}gap> }
\def\QPA{\textsc{Qpa}}
\def\Magma{\textsc{Magma}}

\newcommand{\cat}{{\mathcal C}}
\newcommand{\Ob}{\text{Ob}}
\DeclareMathOperator{\id}{id}
\DeclareMathOperator{\coker}{coker}
\renewcommand{\hom}{\textup{Hom}}
\DeclareMathOperator{\End}{End}
\DeclareMathOperator{\ext}{Ext}
\DeclareMathOperator{\im}{im}
\DeclareMathOperator{\rep}{rep}
\DeclareMathOperator{\SET}{\mathbf{Set}}
\DeclareMathOperator{\CAT}{\mathbf{Cat}}
\DeclareMathOperator{\QUIV}{\mathbf{Quiv}}
\newcommand{\dimvec}[3]{\begin{smallmatrix}#1\\#2\\#3\end{smallmatrix}}


\title{MATH 299Q: Homework 4\\{\Large Quiver Representations}}
\author{}
\date{}

\begin{document}

\maketitle

Let $\k$ be an algebraically closed field.

\begin{enumerate}[1.]
    \item Fix $Q$ to be the Kroenecker $2$-quiver;
    \[\begin{tikzcd}1\arrow[rr,shift left=1,"\alpha"]\arrow[rr,shift right=1,"\beta"']&&2\end{tikzcd}.\]
    Associate to $Q$, the \emph{path algebra} $\k Q$ (recall this is a $\k$-algebra).
    \begin{enumerate}[(a)]
        \item What is $\dim\k Q$?
        \item An \emph{endomorphism} is a morphism from $M$ to itself, that is for some $M\in\rep Q$ the morphism $f:M\to M$. The collection of all endomorphisms, $\End M$, of a fixed representation $M$, much like $\hom(-,-)$, forms a natural $\k$-vector space under composition of morphisms. Let $M$ be the following representation of $Q$;
        \[\begin{tikzcd}\k^2\arrow[rr,shift left=1,"\begin{bmatrix}1&0\\0&1\\0&0\end{bmatrix}"]\arrow[rr,shift right=1,"\begin{bmatrix}0&0\\1&0\\0&1\end{bmatrix}"']&&\k^3\end{tikzcd}.\]
        What is $\dim(\End\k Q)$?
    \end{enumerate}
    \vfill
    \item Fix $Q$ to be as follows;
    \[\begin{tikzcd}
      && && &&2\arrow[lld,"\beta"'] && \\
      6&&5\arrow[ll,"\zeta"']&&4\arrow[ll,"\epsilon"']&& &&1\arrow[llu,"\alpha"']\arrow[lld,"\gamma"] \\
      && && &&3\arrow[llu,"\delta"] &&
    \end{tikzcd}\]
    With the set of relations $R = \{\alpha\beta-\gamma\delta, \beta\epsilon, \delta\epsilon\zeta\}$ generating the admissible ideal $I=\langle R\rangle$. Compute the indecomposable projective representations $\P(1)$, $\P(3)$, and $\P(4)$ of $(Q,I)$.
    \vfill
\end{enumerate}

\end{document}
